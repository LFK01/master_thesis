\documentclass[../../Thesis.tex]{subfiles}

\begin{document}
	Grazie al continuo sviluppo di sistemi robotici cooperativi, la robotica swarm è ormai già una realtà concreta nei sistemi produttivi dei giorni nostri. Guidata dalla continua richiesta di tecnologie affidabili per sistemi stabili e attendibili, l'identificazione di guasti è diventata un campo di ricerca saldo nel tema della robotica swarm. Questa tesi analizza l'importanza delle feature per l'dentificazione di guasti in agenti swarm tramite un metodo centralizzato e data-driven. Questa ricerca si focalizza nel raccogliere una serie di feature semplici e complesse, raggrupparle in tre insiemi e analizzare il loro contributo nella classificazione di anomalie. Il nostro metodo inizia raccogliendo i dati da due simulatori, conosciuti come ARGoS e RAWSim-O, costruisce le serie temporali multivariate per ogni agente ed allena un modello gradient boosting per classificare i campioni anomali da quelli nominali. Per analizzare le performance del modello in relazione ad ogni sottoinsieme di features abbiamo utilizzato le metriche di confusion matrix e preicision recall curve. Per calcolare l'influenza delle feature riguardo la classificazione binaria abbiamo utilizzato la feature permutation importance. In conclusione, vediamo che l'utilizzo di numerose feature o l'aggiunta di feature complesse non migliora notevolmente i risultati. In particolare, le feature più influenti sono quelle più semplici e che catturano le caratteristiche influenzate dal guasto.
	\\
	\\
	\textbf{Parole chiave:} qui, vanno, le parole chiave, della tesi % Keywords (italian)
\end{document}