\documentclass[../../Thesis.tex]{subfiles}

\begin{document}
	
	\section{Results}
		In this work, we have presented a feature importance analysis for fault detection in three different swarm robotic tasks based on the performances of a gradient boosting model. This thesis has the goal to collect different viewpoints on the computation of complex features for fault detection and to analyze the feature influence on achieving the best classification results.\\
		This thesis focuses on centralized fault detection for robotic swarms, the objective is to perform anomaly detection on multivariate time series labeled data with a gradient boosting model and analyze which subset of features has the best performance on binary classification of nominal and faulty robots. Our work is developed on the data collected from two software simulators: ARGoS and RAWSim-O. The ARGoS simulator is used to simulate the flocking and the foraging tasks. The RAWSim-O simulator is used to simulate a warehouse environment with a robot mobile fulfillment system. \\
		The retrieved data is collected from the point of view of an external observer that plays the role of a centralized controller and identifies the malfunctioning agents. To respect the centralized approach, the data collected from the simulators contains only the positions of the robots at each timestep with the corresponding identification number and fault signal flag.\\
		The feature analyzed in out work are the following: position, direction, speed, cumulative speed, neighbors number, neighbors average distance, centroid distance, cumulative centroid distance, position entropy, area coverage, speed area coverage, swarm position, swarm speed, and swarm area coverage. The feature are then grouped in three feature sets that represent three points of view: the first set represents the point of view of the single agent; the second set represents the point of view of the single with the acknowledgment of being surrounded by other robots; the third and last set represents the point of view of the single agent with complete knowledge of its surroundings and the agents of the swarm.\\
		The results we collected reflect our hypothesis made during the feature analysis phase but present also some interesting intuition for more detailed analysis. In general, we can observe that the addition of more attributes and the computation of complex features result in scarce performance improvement and this enhancements may not be worth the additional computation time. As we anticipated, the feature importance is deeply dependent on the type of fault we implement. Since our fault are focused on the speed of the robot, in most cases the speed and cumulative speed feature showed the highest importance values. However, it is interesting to observe that, in the RMFS task, the position entropy feature reaches importance values higher than the speed feature.\\
		In the end we can confirm that, as most of the data-driven fault detection techniques, the results are deeply dependent on the collected data and it is important that the dataset represents the most accurate depiction of real world scenarios. We can observe that adding more features to the dataset is not fundamental and it is important to use as much background knowledge as possible to lighten the load on the data preprocessing procedures and the training phase.
	\section{Future Works}
		The work developed in this thesis can not be considered at all complete and proposes a lot of aspect to further analyze or new procedure to explore.  First of all, the faults analyzed in this thesis do not completely describe real world scenarios and more different or complex fault could be analyzed together with simpler ones. The features subsets proposed in Section \ref{sec:Experiments:gradient_boosting_performance_evaluation} represent a small subset of the possible combinations of features we can analyze, it would be interesting to use smaller subsets with different sets of features and see if it is possible to obtain significant result.Nonetheless, we have to consider all the possible different combinations of parameters, to list some: the \verb|time-window| and \verb|neighborhood-radius|, the settings of the swarm robots' tasks, the parameters of the gradient boosting model and so on.\\
		For what it regards the model used to perform fault detection, it presents a lot of chances for improvements. The model used in our work does not exploit time series time-related characteristics and does not remember when a robot is faulty. The approach used for fault detection could be reinterpreted in order to identify the timestep in which the fault gets injected and measure the delay between the real injection and the predicted injection. Furthermore, we could use deep learning models that analyze a time window of features of size $k \times d$, with $k$ as the number of timesteps in the past and $d$ as the number of features. This new model would interpret a number $t$ of features windows in a convolutional-LSTM model to exploit the historical characteristic of the time series.\\
		From a broader view it would be interesting to analyze the importance of the features on the activity of task conclusion detection. In some scenarios, where it is necessary to have an external expert that identifies when the swarm robotic task has fulfilled its objective, it would be interesting to see which features are more discriminant in detecting the task complete execution.\\
\end{document}