\documentclass[../../Thesis.tex]{subfiles}

\begin{document}
	With the blooming of the cooperation among robotic systems, swarm robotic has become a concrete reality in nowadays production methodology. Driven by the never ending demand of reliable technologies for stable and trustworthy systems, fault detection has become a firm field of study in the topic of swarm robotics. This thesis analyzes the importance of feature for fault detection in swarm agents using a data-driven centralized approach. The focus of this work is to collect a list of simple and complex feature, group them in three sets and analyze their contribution for the classification of anomalies. Our approach starts with the data collected from the two simulators, namely ARGoS and RAWSim-O, computes the multivariate time series for each agent and trains a gradient boosting model to classify nominal and faulty samples. The performance of the model are analyzed on the three sets of features using confusion matrices and precision recall curves. The feature influence on the binary classification task is computed using the feature permutation importance. In the, end we can see that the presence of more complex features does not improve significantly the model performances. More importantly, the most influential features are the ones that are simpler and that are related to the type of injected fault.
	\\
	\\
	\textbf{Keywords:} swarm robotics, fault detection, feature importance, flocking, foraging % Keywords
\end{document}